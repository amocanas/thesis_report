\section*{Introduction}
Those first couple of years after university are an intensely action-packed, exciting time. The students have finished school,they’ve made plans to move out of their parents’ house, and they’ve started a career and are getting paid.Maybe not as much as they’d like, and maybe not as much as some of their friends are making, but they’re finally starting to see some cash roll in.Actually here is the beginning of that delicate time, beginning of the life as an adult, and it is the moment when you are finally in control of your money.The beginning of the financial life is no less delicate, and your wealth in your thirties and forties will be mainly determined by the financial decisions make right after university. A common problem encountered by young people at this time is wrong money management, this is a time when the desire for independence combined with optimism about the future drives everything.So, here appears the need of having more control over money, and a mobile application which will scan all kind of receipts and will make a kind of statistics of all expenditures accumulated may be will be the best choice of everyone who will care about the tracking budget correctly.

At the moment there a lot of applications helping you to track your expenditures but all of them require a manual input of all outgoings, which is somehow uncomfortable and requires more time to do. Also, given the fact that today's young generation is increasingly lazy and is always looking for easier solutions then such kind of application seems to be very attractive for them. Moreover it will be convenient to use it as well for older people, such as it's user interface will is very clear and simple to use for all groups of people in the society.

The main purpose of the Spending Tracker is scanning the receipts and showing a total amount spend by selecting a desired period,besides this it has a lot of other specific functionalities, such as displaying a statistic of expenditures by categories of stuff,categorizing expenditures by food, drinks, communal services, entertainment, fuel and others or categorizing by stores,advertising user when a product is cheaper in one store that in other,also it's possible to set a reminder of upcoming payment which serves people as a strong point in making themselves more responsible.

For users to access the application is necessary to install it on an Android platform, to open it and enjoy the user friendly interface which allows user to fast obtain a result.The first screen is a Welcome screen which informs users about the main functionalities of the application and get the user ready to start using it.Then the main fragment is opened which contains the total amount of expenditures for today by default, and the list of them.For scanning the receipt another action is required,by pressing the plus button there is opened another view which informs you about the process of scanning the receipt,for that is necessary to take a photo of the receipt,then all necessary information is stocked in a database for forward use.The application contains a spinner in the main fragment where can be fount all features: Spendings,Categories,Statistics and some History which includes the list with all the stuff and the amounts of them inside. 

The current thesis consists of 4 chapters. Chapter 1 explains the application analysis, together with the general overview about the final product. A market research has been done, in order to define advantages/disadvantages of using the application and provide a better understanding over the resulting system. Chapter 2 contains a more complex description of the application, from architectural point of view. The application is examined from different approaches, using UML diagrams. Chapter 3 contains a brief explanation regarding the technologies used for the implementation process and how those technologies were linked to the application.Chapter 4 represents a review over the economical position of the product. It is presented an estimation of costs for developing the project and some economical prediction for the future use.
