\section{ Analysis of the application from architectural point of view}
In the requirements phase, earlier,was built an abstract model of the problem,which identifies what objects,entities,views are involved,how they relate to one another and how they look like.Probably now is the moment to describe the software architecture more thoroughly, from the implementation point of view,describing all the most important interfaces,classes, interaction between user and system and last but not least - the interaction between system components.In order to characterize the Spending Tracker application system, a collection of models were used:
\begin{itemize}
	\item \textbf{Use Case model}.It represents a set of use cases, actors, and their relationships.A use case represents a particular functionality of a system and it is used to describe the relationships among the functionalities and their controllers known also as \textbf{actors}.
	\item \textbf{Activity model}.It describes the flow of control in a system. It consists of activities and links.Activities are nothing but the functions of a system. Numbers of activity diagrams are prepared to capture the entire flow in a system.
	\item \textbf{Sequence model}.A sequence diagram is an interaction diagram.It is clear that the diagram deals with some sequences, which can be sequence of messages flowing from one object to another.
	\item \textbf{ Database model}.It represents a static structure diagram which underlines which are the entities where data are stored in the system,and the relationships which exist between them. \cite{UML}
\end{itemize}

This is the chapter where the Spending Tracker application is analysed using the above described diagrams.This will help to get a more clear image and a better understanding about the entire system and about the components interaction.

\subsection{Representation of the system via Use Case Diagrams}
\subsection{System Analysis using Activity Diagrams}
\subsection{Process interaction analysis using Sequence Diagram}
\subsection{The Database Model of the system}
\subsection{Final stage of product development: Deployment Diagram}