\section{Economic Analysis}
\subsection{Project description}
Spending Tracker is an application developed for different target groups,for young people which face problems at the chapter of correct money management and as well for older people,who want to monitor all expenses,such as it's user interface is very clear and simple to use for all groups of people in the society.The main task of the application is scanning the receipts and displaying the total expenses for different periods,depending on user's choice,and for different stores,which present categories in the statistics bar chart.Such as it does not require a manual input introducing, it presents a major advantage for its users.This application ensures managements of two most important values in the life: money and time.It saves time by offering that simple possibility of scanning the receipt, and it saves money of course because the main purpose of the application is tracking right the budget,by monitoring day by day all the expenses,and making some decisions according the correctness of their spending.

It is worth mentioning that such applications already exist on the market but they're lack of the most functionality the Spending Tracker has,automatic receipt scanner by taking a photo of the receipt.Also,it's strong point is the simplicity in use,it does not require connection to a bank account or other dubious actions as can be observed in other existing applications on market.It offers exactly what needs every person,nothing more,a user friendly interface and a quick result displaying.

Before proceeding to the implementation of the system,it is necessary to analyse the project budget,this will help to find a justifiable economical point of view over the system.In order to understand and decide if the product is workable on the national market,it would be better to elaborate an overview of the expenditures and incomes after releasing the application.Given that fact,the economic analysis should be one of the first steps in starting development of any product,whose final result would be a general overview on the elaboration of the system.

This chapter represents the analysis of all expenses necessary to elaborate the application,starting from  materials/non-materials used for long term, time schedule establishment and indirect expenses. After that,the days necessary for development should be estimated,should be established the working team and of course each team member salary.Only after this starting analysis will be clear what steps should be performed further in order to have a gain.Even this chapter is last,it is in fact the most important one in real life,because economic analysis is a means to help bring about a better allocation of resources that can lead to enhanced incomes for investment or consumption purposes,it is best undertaken at the early stages of the project cycle to enable decision makers to make an informed decision on whether to undertake a particular investment given various alternatives and their corresponding costs.

\newpage
\subsection{Project time schedule}
Taking into account that the Spending Tracker application is quite complex and have in the requirements list many features to be implemented,it would be better to plan a time schedule.In order to get the better results faster,and also a considerable income,will be tried to be used \textbf{Agile Software Development}.In such a way the product will be delivered faster,maybe not implemented at all,a process of frequent feedbacks will be adopted for excluding risks of a single-pass development when all the requirements are implemented.The future modifications can be treated as improvements for the base application which represents a benefit of Agile.
\subsubsection{SWOT Analysis}
\subsubsection{Defining objectives}
\subsubsection{Time schedule establishment}
\subsection{Economic motivation}
\subsubsection{Tangible and intangible asset expenses}
\subsubsection{Salary expenses}
\subsection{Individual person salary}
\subsubsection{Indirect expenses}
\subsubsection{Wear and depreciation}
\subsubsection{Product cost}
\subsubsection{Economic indicators and results}
\subsection{Economic conclusions}